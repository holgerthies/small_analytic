%% FIRST RENAME THIS FILE <yoursurname>.tex. 
%% 
%% TO PROCESS THIS FILE YOU WILL NEED TO DOWNLOAD 
%% llncs.cls and remreset.sty from 
%% ftp://ftp.springer.de/pub/tex/latex/llncs/latex2e/llncs2e.zip
%% This abstract should be 1page.

\documentclass{llncs}
\usepackage{amsmath,amssymb}
\newcommand{\DD}{\mathbb D}
\usepackage{xspace}
\newcommand{\RR}{\mathbb R}
\newcommand{\CC}{\mathbb C}
\DeclareMathOperator{\NN}{\mathbb N}
\DeclareMathOperator{\C}{\mathcal C}
\DeclareMathOperator{\laplace}{\Delta}
\newcommand{\abs}[1]{\left|#1\right|}
\newcommand{\p}{\ensuremath{\mathcal P}\xspace}
\newcommand{\np}{\ensuremath{\mathcal{NP}}\xspace}
\newcommand{\cl}{\ensuremath{\mathcal{L}}\xspace}
\newcommand{\nc}{\ensuremath{\mathcal{NC}}\xspace}
\newcommand{\fp}{\ensuremath{\mathcal{FP}}\xspace}
\newcommand{\sharpp}{\ensuremath{\# \mathcal{P}}\xspace}
\newcommand{\pspace}{\ensuremath{ \mathcal{PSPACE}}\xspace}
\newcommand{\cc}{\texttt{C++}\xspace}
\newcommand{\irram}{\texttt{iRRAM}\xspace}

\begin{document}

\title{Analytic Functions and Small Complexity Classes \thanks{The authors thank the JSPS Core-to-Core Program (Advanced Research Networks) and Grants-in-Aid for Scientific Research (Kakenhi 24106002 and 26700001) for supporting the research presented here. The second author was supported by the International Research Training Group 1529 ‘Mathematical Fluid Dynamics’ funded by the DFG and JSPS.}}

%% SINGLE AUTHOR
%\author{FirstName LastName}
%\institute{Department, University, Street Address, Country,
%\email{email address}}

%% MULTIPLE AUTHORS
%%% NOTE: email required for at least one author
\author{Akitoshi Kawamura\inst{1} \and Florian Steinberg\inst{2} \and Holger Thies\inst{1} }
\institute{Graduate School of Arts and Sciences, The University of Tokyo, 3-8-1 Komaba, Meguro-ku, Tokyo 153-8902, Japan, kawamura@graco.c.u-tokyo.ac.jp, holgerthies@g.ecc.u-tokyo.ac.jp  \and Department of Mathematics, TU Darmstadt, Schlo\ss gartenstra\ss e 7, 64289 Darmstadt, Germany, \email{steinberg@mathematik.tu-darmstadt.de}}

\maketitle

%% INSERT TEXT OF ABSTRACT DIRECTLY BELOW
Kawamura and Cook's extension to Weihrauch's framework of representations \cite{Weihrauch} gives a reasonable model to analyze the complexity of operators in analysis \cite{AkiACM}.
Defining type-2 versions of the complexity classes \p, \np and \pspace, they generalized many non-uniform hardness results into uniform statements.

Using this model, it can be shown that many important operators are in general hard from the viewpoint of real complexity theory. 
Nevertheless, restricting to only analytic functions many of these operators can be computed in polynomial time \cite{MR1137517}.
Using the right representation, those statements can be turned into uniform algorithms \cite{Kawamura2012}.
Since applying operators to power series often involves modifications on a large number of coefficients at the same time, it is reasonable to ask which operations on analytic functions can be efficiently parallelized.

In classical complexity theory, efficient parallelization is strongly connected to the complexity classes \cl and \nc.
A generalization of this in the sense of continuous problems was recently done by Kawamura and Ota \cite{Kawamura2014}.
Using their framework, we generalize the polynomial time results on analytic functions and investigate which operations are eligible for efficient parallelization and which most likely are not.


\begin{thebibliography}{10}
\providecommand{\url}[1]{\texttt{#1}}
\providecommand{\urlprefix}{URL }


\bibitem{AkiACM}
  Kawamura, A., Cook, S.,
  {\it Complexity Theory for Operators in Analysis},
  ACM Transactions in Computation Theory,
vol.~4 (2012), no.~2, pp.~5:1--5:24.

\bibitem{Kawamura2012}
  Kawamura, A., M{\"u}ller, N.T., R{\"o}snick, C., Ziegler, M.,
  {\it Computational benefit of smoothness: Parameterized bit-complexity of numerical operators on analytic functions and Gevrey’s hierarchy },
 Journal of Complexity, 
vol.~31, no.~5, pp.~680--714.

\bibitem{Kawamura2014}
  Kawamura, A., Ota, H.,
  {\it Small complexity classes for computable analysis.},
  Lecture Notes in Computer Science,
  vol.~8635,  pp.~432--444.

\bibitem{MR1137517}
  Ko, K., Friedman, H.
  {\it Computational complexity of real functions},
Theoretical Computer Science,
vol.~20 (1982), no.~3, pp.~323--352.

\bibitem{Weihrauch}
Weihrauch, K.
{\it Computable Analysis, an Introduction},
Springer,
2000.
\end{thebibliography}
\end{document}
