\documentclass{article}
\usepackage[margin=1.1in]{geometry}
\usepackage[utf8]{inputenc}
\usepackage{xspace}
\usepackage{amsmath,amssymb}
\usepackage{amsthm}
\newcommand{\DD}{\mathbb D}
\newcommand{\RR}{\mathbb R}
\newcommand{\CC}{\mathbb C}
\DeclareMathOperator{\NN}{\mathbb N}
\DeclareMathOperator{\C}{\mathcal C}
\DeclareMathOperator{\laplace}{\Delta}
\bibliographystyle{amsalpha}
\newcommand{\abs}[1]{\left|#1\right|}
\newcommand{\p}{\ensuremath{\mathcal P}\xspace}
\newcommand{\np}{\ensuremath{\mathcal{NP}}\xspace}
\newcommand{\cl}{\ensuremath{\mathcal{L}}\xspace}
\newcommand{\nc}{\ensuremath{\mathcal{NC}}\xspace}
\newcommand{\fp}{\ensuremath{\mathcal{FP}}\xspace}
\newcommand{\sharpp}{\ensuremath{\# \mathcal{P}}\xspace}
\newcommand{\pspace}{\ensuremath{ \mathcal{PSPACE}}\xspace}
\newcommand{\cc}{\texttt{C++}\xspace}
\newcommand{\irram}{\texttt{iRRAM}\xspace}
\newtheorem{definition}{Definition}
\newtheorem{theorem}{Theorem}
\begin{document}
\section*{Analytic Functions and Small Complexity Classes}
Kawamura and Cook's extension to TTE \cite{AkiACM} gives a reasonable model to  analyze the complexity of operators in analysis.
Defining type-2 versions of the complexity classes \p, \np and \pspace, they could generalize many non-uniform hardness results into uniform statements.

Analytic functions not only play a central role in analysis but also have been thoroughly studied in real computability and complexity theory.
One reason being that many operations that are hard on general computable functions become feasible when only considering analytic functions.
E.g., integration is \sharpp-hard in general \cite{MR748898,AkiACM} but polynomial-time computable for analytic functions \cite{MR1137517, Kawamura2012}.  

Let $D \subseteq \CC$ be the closed unit disc and consider functions analytic on $D$.
It is well known, that such a function is computable if and only if the power series around every computable point in its domain is computable \cite{MR1137517}.
The same holds for polynomial-time computability.
However, those results are non-uniform in the sense that the operations of obtaining the power series from a function and obtaining the function from the power series are non-computable \cite{Mueller95}.
For power series around $0$ with radius of convergence strictly larger than $1$, we therefore use the following representation  as considered e.g. in \cite{Kawamura2012}.
\begin{definition}
  A (length-monotone) function $\varphi: \Sigma^* \to \Sigma^*$ is a name for the power series $(a_k)_{k \in \NN}$ iff
  $\varphi(0)$ encodes two integers $A$ and $k$,
  the function $n \to \varphi(n+1)$ is a name for the sequence $(a_k)_{k \in \NN}$ and
  $\abs{a_n} \leq A \cdot 2^{-\frac{n}{k}}$
\end{definition}
Under this representation many operators like evaluation, multiplication or differentiation of power series become polynomial time-computable \cite{Kawamura2012}.
Additionally, solving ordinary differential equations with analytic right-hand side is polynomial time computable \cite{Kawamura10}.
We want to refine those complexity results in terms of small complexity classes, i.e., complexity classes contained in \p.
Such classes were recently defined by Kawamura and Ota \cite{Kawamura2014}.
In particular they defined type-two anologues of the classes \cl and \nc and \p-completeness.
\begin{theorem}
  The operator Sum: $\mathcal{O} \times [-1,1] \to \RR, \text{Sum}(a)(x) = \sum_{n=0}^\infty a_nx^n$ is log-space computable.  
\end{theorem}

\bibliography{bib}{}

\end{document}

