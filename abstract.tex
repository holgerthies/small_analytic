\documentclass{article}
\usepackage[margin=1.1in]{geometry}
\usepackage[utf8]{inputenc}
\usepackage{xspace}
\usepackage{amsmath,amssymb}
\usepackage{amsthm}
\newcommand{\DD}{\mathbb D}
\newcommand{\RR}{\mathbb R}
\newcommand{\CC}{\mathbb C}
\DeclareMathOperator{\NN}{\mathbb N}
\DeclareMathOperator{\C}{\mathcal C}
\DeclareMathOperator{\laplace}{\Delta}
\bibliographystyle{amsalpha}
\newcommand{\abs}[1]{\left|#1\right|}
\newcommand{\p}{\ensuremath{\mathcal P}\xspace}
\newcommand{\np}{\ensuremath{\mathcal{NP}}\xspace}
\newcommand{\cl}{\ensuremath{\mathcal{L}}\xspace}
\newcommand{\nc}{\ensuremath{\mathcal{NC}}\xspace}
\newcommand{\fp}{\ensuremath{\mathcal{FP}}\xspace}
\newcommand{\sharpp}{\ensuremath{\# \mathcal{P}}\xspace}
\newcommand{\pspace}{\ensuremath{ \mathcal{PSPACE}}\xspace}
\newcommand{\cc}{\texttt{C++}\xspace}
\newcommand{\irram}{\texttt{iRRAM}\xspace}
\newtheorem{definition}{Definition}
\newtheorem{theorem}{Theorem}
\begin{document}
\section*{Analytic Functions and Small Complexity Classes}
Kawamura and Cook's extension to TTE \cite{AkiACM} gives a reasonable model to  analyze the complexity of operators in analysis.
Defining type-2 versions of the complexity classes \p, \np and \pspace, they could generalize many non-uniform hardness results into uniform statements.

Analytic functions not only play a central role in analysis but also have been thoroughly studied in real computability and complexity theory.
One reason being that many operations that are hard on general computable functions become feasible when only considering analytic functions.
E.g., integration is \sharpp-hard in general \cite{MR748898,AkiACM} but polynomial-time computable for analytic functions \cite{MR1137517, Kawamura2012}.  

Let $D \subseteq \CC$ be the closed unit disc and consider functions analytic on $D$.
It is well known, that such a function is computable if and only if the power series around every computable point in its domain is computable \cite{MR1137517}.
The same holds for polynomial-time computability.
However, those results are non-uniform in the sense that the operations of obtaining the power series from a function and obtaining the function from the power series are non-computable \cite{Mueller95}.
A more reasonable representation for analytic functions is e.g. considered in \cite{Kawamura2012}.
\begin{definition}\label{def:function}
  A (length-monotone) function $\varphi: \Sigma^* \to \Sigma^*$ is a name for a complex analytic function $f:D \to \CC$ iff
  $\varphi(0)$ is an integer $A$ encoded in binary,
  $\varphi(1)$ is an integer $k$ encoded in unary,
  the function $n \to \varphi(n+2)$ is a name for the function $f$,
  $f$ extends analytically to $B(0, \sqrt[k]{2})$ and
  $\abs(f(z)) \leq A$ for all $z \in B(0, \sqrt[k]{2})$

\end{definition}
Similarly one can define a representation for power series 
\begin{definition}\label{def:powerseries}
  A (length-monotone) function $\varphi: \Sigma^* \to \Sigma^*$ is a name for a power series $(a_k)_{k \in \NN}$ iff
  $\varphi(0)$ is an integer $A$ encoded in binary,
  $\varphi(1)$ is an integer $k$ encoded in unary,
  the function $n \to \varphi(n+2)$ is a name for the sequence $(a_k)_{k \in \NN}$ and
  $\abs{a_n} \leq A \cdot 2^{-\frac{n}{k}}$ for all $n \in \NN$.
\end{definition}
Those two representations can be shown to be polynomial time equivalent  and many operations on analytic functions like differentiation or parametric maximization become polynomial time computable in this representation \cite{Kawamura2012}.

Our goal is to refine those results by considering complexity classes inside \p.
Such classes were recently defined by Kawamura and Ota \cite{Kawamura2014}.
In particular they defined type-two anologues of the classes \cl and \nc and a notion \p-completeness.
Using their definition, we can show that the operator summing up power series (according to Definition \ref{def:powerseries}) as well as the operator evaluating the $d$-th derivative of a function given according Definition \ref{def:function} are in \cl.
It follows that the two representations are logspace equivalent.
We further consider other operations on analytic functions and how they relate to small complexity classes.

\bibliography{bib}{}

\end{document}

